\RequirePackage[l2tabu, orthodox]{nag}
%\documentclass[a4paper,12pt]{article}
\documentclass[man,floatsintext]{apa6} % man, doc, jou

\usepackage{amsmath}
% \usepackage[a4paper]{geometry}
\usepackage{graphicx}
\usepackage{microtype}
\usepackage{siunitx}
\usepackage{booktabs}
\usepackage{apacite}
% \usepackage{natbib}
%\usepackage{newclude} % \include* to include without page break
%\usepackage[colorlinks=false, pdfborder={0 0 0}]{hyperref}
\usepackage{cleveref} % \cref instead of \ref, automatic "Section" etc.
\usepackage{eurosym}

\title{Predicting reaction time using oscillations in the beta band}
\shorttitle{Predicting reaction time with beta activity}
\author{Max Hollmann}
\affiliation{University of Groningen}
\date{\today}
\authornote{Thanks to Hedderik van Rijn, Mark Span, Udo Boehm, and Jacob Jolij.}

\abstract{
  Abstract...
}
\keywords{beta power, interval timing, reaction time}

\begin{document}
\maketitle
% TODO no newline between \maketitle and Intro.

%% Hedderik: Slow potentials...
% History of area (mostly specific stuff (CNV in combination with timing))
% Explain terms (What's CNV?)
% Prior research on topic
% Explanations/mechanisms given by prior papers
% New explanation/mechanism from other research
%

% Hedderik in the method:
% The Method section is pretty technical, which is perfectly fine.
% However, this does mean that you will need to explain the rationale of
% the setup of the experiment (i.e., waiting period, then cue and as
% fast as possible reaction) in the introduction

%%% Gambit: Time perception, timing, etc
%%% Neural mechanism of timing?
%%% CNA/CNV
% CNA: climbing neural activity
% CNV: negative buildup of EEG "when subject is expecting an event"
%   proposed to reflect the accumulation part of the pacemaker-accumulator models of timing
%   Macar et al 1999: larger cnv for longer produced intervals (also for longer judged intervals)
%   Not reproduced by Kononowitz & Van Rijn 2011
%%% Beta in tapping experiments (Bartolo, Joundi)
% Beta power at beginning of interval predicts length of produced interval in synchronization-continuation tapping experiment
%%% Beta in interval production (Kononowitz & van Rijn in prep)
% Beta power at beginning of interval predicts length of produced interval
%%% Interval timing in reaction time (Grosjean? or something similar)
% Interval timing plays a role in reaction time.
%%% (Internal/external cues)

The passage of time is one of the most basic facts of our world, making timing one of the most important functions of the mind.
Thus, a lot of scientific effort is spent on discovering its underlying mechanisms.

% CNV -> interval length
One line of research in this area is looking at connections between electrophysiological measures from the brain and interval timing.
For example, \citeA{macar_supplementary_1999} found a positive correlation between the amplitude of a negative cortical potential during a self-produced interval and the duration of the same interval.
However, in a replication of this experiment \citeA{kononowicz_slow_2011} failed to find the same effect.
% This idea, expressed in the work of Durstewitz (2003, 2004),
% does not predict amplitude differences for short and long produc-
% tions, and has found support in timing paradigms
% Durstewitz, D. (2003). Self-organizing neural integrator predicts interval times through climbing activity. J. Neurosci. 23, 5342–5353.
% Durstewitz, D. (2004). Neural representation of interval time. Neuroreport 15, 745–749.

% Beta -> interval length
% tapping -> off/onset same -> no predictability

More recently, a different measure was found to be predictive of the length of produced intervals.
\citeA{bartolo_information_2014} used the power of neuronal oscillations in the beta band to predict the interval length in a synchronization-continuation tapping task performed on monkeys.
In this experiment, subjects first tapped along with an externally provided beat (synchronization), and then continued tapping at the same speed without external guidance (continuation).
During the continuation part, longer intervals were associated with higher beta power during the interval, and vice versa.
This makes sense in the light of previous findings that linked beta power to motor inhibition \cite{joundi_driving_2012}.

\citeA{kononowicz_beta_2014} extended these findings to an interval production task, in which participants were instructed to estimate time intervals of 2.5 seconds by pressing a key twice.
In this study, intervals were distinct from each other, allowing a clear association between the beta power during the interval and the length of the interval itself.
As in \citeauthor{bartolo_information_2014}'s study, beta power was postitively correlated with the length of the produced intervals.

% Interval timing -> reaction time
In another area of the field of timing, a large body of research exists that links interval timing to reaction time tasks \cite<e.g.,>{karlin_reaction_1959, drazin_effects_1961, grosjean_timing_2001}.
When participants come to expect the trigger stimulus after a specific time interval, deviations from this interval lead to predictable changes in reaction time.
For example, \citeA{grosjean_timing_2001} found that when the stimulus appears earlier than expected, reaction times will be slower.
Conversely, when it appears later than expected, reaction times will be faster.
An obvious explanation for this effect is that participants build up the readiness to react towards the point in time when they expect the stimulus to occur.
When it appears before they expect it, they are less ready to respond to it, and thus slower.
However, when it appears later than anticipated, the readiness is very high and the reaction comes faster.

% This study
The purpose of the present study is to investigate the link between beta power, interval timing, and reaction time.

\section{Method}
\section{Results}

% TODO update beta coeffs with all electrodes
\begin{table}[ht]
  \caption{Regression of reaction time on length condition and beta power}
  \centering
  \begin{tabular}{lrrrr}
    \toprule
                                                  & B        & SE         & t       & p         \\
    \midrule
    Intercept                                     & -0.0029  & 0.0008     &  -3.42  & $< 0.001$ \\
    Length condition: long                        & -0.0171  & 0.0016     & -10.44  & $< 0.001$ \\
    Length condition: short                       &  0.0259  & 0.0015     &  16.96  & $< 0.001$ \\
    Beta power                                    & -0.0031  & 0.0063     &  -0.49  & $0.622$   \\
    Beta power $\times$ Length condition: long    & -0.0079  & 0.0123     &  -0.64  & $0.521$   \\
    Beta power $\times$ Length condition: short   & -0.0086  & 0.0130     &   0.66  & $0.511$   \\
    \bottomrule
  \end{tabular}
\end{table}

\begin{table}[ht]
  \caption{Descriptive statistics overall as well as by length condition}
  \centering
  \begin{tabular}{lrrrrr}
    \toprule
                              & \multicolumn{2}{l}{Reaction time}              && \multicolumn{2}{l}{Beta}                       \\ \cmidrule{2-3} \cmidrule{5-6}
                              & \multicolumn{1}{l}{M} & \multicolumn{1}{l}{SD} && \multicolumn{1}{l}{M} & \multicolumn{1}{l}{SD} \\
    \midrule
    Overall                   & 0.188                 &  0.067                 && 0.277                 & 0.188                  \\
    Length condition: short   & 0.207                 &  0.070                 && 0.270                 & 0.168                  \\
    Length condition: medium  & 0.188                 &  0.061                 && 0.283                 & 0.200                  \\
    Length condition: long    & 0.166                 &  0.070                 && 0.271                 & 0.177                  \\
    \bottomrule
  \end{tabular}
\end{table}

\section{Discussion}


\bibliographystyle{apacite} % / apalike
% http://www.ctan.org/tex-archive/biblio/bibtex/contrib/apacite/apacite.pdf
\bibliography{/home/max/Dropbox/Psychology/Thesis/latex/clean_references}


\end{document}

\end{document}
