\RequirePackage[l2tabu, orthodox]{nag}
%\documentclass[a4paper,12pt]{article}
\documentclass[man]{apa6} % man, doc, jou

\usepackage{amsmath}
% \usepackage[a4paper]{geometry}
\usepackage{graphicx}
\usepackage{microtype}
\usepackage{siunitx}
\usepackage{booktabs}
\usepackage{apacite}
% \usepackage{natbib}
%\usepackage{newclude} % \include* to include without page break
%\usepackage[colorlinks=false, pdfborder={0 0 0}]{hyperref}
\usepackage{cleveref} % \cref instead of \ref, automatic "Section" etc.
\usepackage{eurosym}

\title{Predicting reaction time using oscillations in the beta band}
\shorttitle{Predicting reaction time with beta activity}
\author{Max Hollmann}
\affiliation{University of Groningen}
\date{\today}
\authornote{Thanks to Hedderik van Rijn, Mark Span, Udo Boehm, and Jacob Jolij.}

\abstract{
  Abstract...
}
\keywords{beta power, interval timing, reaction time}

\begin{document}
\maketitle
% TODO no newline between \maketitle and Intro.

%% Hedderik: Slow potentials...
% History of area (mostly specific stuff (CNV in combination with timing))
% Explain terms (What's CNV?)
% Prior research on topic
% Explanations/mechanisms given by prior papers
% New explanation/mechanism from other research
%

%%% Gambit: Time perception, timing, etc
%%% Neural mechanism of timing?
%%% CNA/CNV
% CNA: climbing neural activity
% CNV: negative buildup of EEG "when subject is expecting an event"
%   proposed to reflect the accumulation part of the pacemaker-accumulator models of timing
%   Macar et al 1999: larger cnv for longer produced intervals (also for longer judged intervals)
%   Not reproduced by Kononowitz & Van Rijn 2011
%%% Beta in tapping experiments (Bartolo, Joundi)
% Beta power at beginning of interval predicts length of produced interval in synchronization-continuation tapping experiment
%%% Beta in interval production (Kononowitz & van Rijn in prep)
% Beta power at beginning of interval predicts length of produced interval
%%% Interval timing in reaction time (Grosjean? or something similar)
% Interval timing plays a role in reaction time.
%%% (Internal/external cues)

The passage of time is one of the most basic facts of our world, making timing one of the most important functions of the mind.
Thus, a lot of scientific effort is spent on discovering its underlying mechanisms.
One promising line of research in this area is looking at connections between electrophysiological measures from the brain and interval timing.
For example, \citeA{macar_supplementary_1999} found a positive correlation between the amplitude of a negative cortical potential during a self-produced interval and the duration of the same interval.
However, a replication of this experiment by \citeA{kononowicz_slow_2011} failed to find the same effect.

More recently, a different measure was found to be predictive of the length of produced intervals.
\citeA{bartolo_information_2014} used the power of neuronal oscillations in the beta band to predict the interval length in a synchronization-continuation tapping task.
In this experiment, participants first tapped along with an externally provided beat (synchronization), and then continued tapping at the same speed without external guidance (continuation).
During the continuation part, longer intervals were associated with higher beta power during the interval, and vice versa.
This makes sense in the light of previous findings that linked beta power to motor inhibition \cite{joundi_driving_2012}.


\cite{kononowicz_beta_nodate}.

\section{Method}
\section{Results}

% TODO update beta coeffs with all electrodes
\begin{table}[ht]
  \caption{Regression of reaction time on length condition and beta power}
  \centering
  \begin{tabular}{lrrrr}
    \toprule
                                                  & B        & SE         & t       & p         \\
    \midrule
    Intercept                                     & -0.0029  & 0.0008     & -3.41   & $< 0.001$ \\
    Length condition: long                        & -0.0171  & 0.0016     & -10.47  & $< 0.001$ \\
    Length condition: short                       & 0.0259   & 0.0015     & 16.95   & $< 0.001$ \\
    Beta power                                    & 0.0006   & 0.0015     & 0.41    & $0.683$   \\
    Beta power $\times$ Length condition: long    & -0.0047  & 0.0030     & -1.55   & $0.121$   \\
    Beta power $\times$ Length condition: short   & -0.0001  & 0.0031     & -0.02   & $0.980$   \\
    \bottomrule
  \end{tabular}
\end{table}

\begin{table}[ht]
  \caption{Descriptive statistics overall as well as by length condition}
  \centering
  \begin{tabular}{lrrrrr}
    \toprule
                              & \multicolumn{2}{l}{Reaction time}              && \multicolumn{2}{l}{Beta}                       \\ \cmidrule{2-3} \cmidrule{5-6}
                              & \multicolumn{1}{l}{M} & \multicolumn{1}{l}{SD} && \multicolumn{1}{l}{M} & \multicolumn{1}{l}{SD} \\
    \midrule
    Overall                   & .188                  &  .067                  && .277                  & .188                   \\
    Length condition: short   & .207                  &  .070                  && .270                  & .168                   \\
    Length condition: medium  & .188                  &  .061                  && .283                  & .200                   \\
    Length condition: long    & .166                  &  .070                  && .270                  & .177                   \\
    \bottomrule
  \end{tabular}
\end{table}

\section{Discussion}


\bibliographystyle{apacite} % / apalike
% http://www.ctan.org/tex-archive/biblio/bibtex/contrib/apacite/apacite.pdf
\bibliography{/home/max/Dropbox/Psychology/Thesis/latex/clean_references}


\end{document}

\end{document}
