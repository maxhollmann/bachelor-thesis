\RequirePackage[l2tabu, orthodox]{nag}
%\documentclass[a4paper,12pt]{article}
\documentclass[man,floatsintext]{apa6} % man, doc, jou

\usepackage{amsmath}
% \usepackage[a4paper]{geometry}
\usepackage{graphicx}
\usepackage{microtype}
\usepackage{siunitx}
\usepackage{booktabs}
\usepackage{apacite}
% \usepackage{natbib}
%\usepackage{newclude} % \include* to include without page break
%\usepackage[colorlinks=false, pdfborder={0 0 0}]{hyperref}
\usepackage{cleveref} % \cref instead of \ref, automatic "Section" etc.
\usepackage{eurosym}

\title{Predicting reaction time using oscillations in the beta band}
\shorttitle{Predicting reaction time with beta activity}
\author{Max Hollmann}
\affiliation{University of Groningen}
\date{\today}
\authornote{Thanks to Hedderik van Rijn, Mark Span, Udo Boehm, and Jacob Jolij.}

\abstract{
  Abstract...
}
\keywords{beta power, interval timing, reaction time}

\begin{document}
\maketitle
% TODO no newline between \maketitle and Intro.

%% Hedderik: Slow potentials...
% History of area (mostly specific stuff (CNV in combination with timing))
% Explain terms (What's CNV?)
% Prior research on topic
% Explanations/mechanisms given by prior papers
% New explanation/mechanism from other research
%

% Hedderik in the method:
% The Method section is pretty technical, which is perfectly fine.
% However, this does mean that you will need to explain the rationale of
% the setup of the experiment (i.e., waiting period, then cue and as
% fast as possible reaction) in the introduction

%%% Gambit: Time perception, timing, etc
%%% Neural mechanism of timing?
%%% CNA/CNV
% CNA: climbing neural activity
% CNV: negative buildup of EEG "when subject is expecting an event"
%   proposed to reflect the accumulation part of the pacemaker-accumulator models of timing
%   Macar et al 1999: larger cnv for longer produced intervals (also for longer judged intervals)
%   Not reproduced by Kononowitz & Van Rijn 2011
%%% Beta in tapping experiments (Bartolo, Joundi)
% Beta power at beginning of interval predicts length of produced interval in synchronization-continuation tapping experiment
%%% Beta in interval production (Kononowitz & van Rijn in prep)
% Beta power at beginning of interval predicts length of produced interval
%%% Interval timing in reaction time (Grosjean? or something similar)
% Interval timing plays a role in reaction time.
%%% (Internal/external cues)

The passage of time is one of the most basic facts of our world, making timing one of the most important functions of the mind.
Thus, a lot of scientific effort is spent on discovering its underlying mechanisms.

% CNV -> interval length
One line of research in this area is looking at connections between electrophysiological measures from the brain and interval timing.
For example, \citeA{macar_supplementary_1999} found a positive correlation between the amplitude of a negative cortical potential during a self-produced interval and the duration of the same interval.
However, in a replication of this experiment \citeA{kononowicz_slow_2011} failed to find the same effect.

% Beta -> interval length
% tapping -> off/onset same -> no causality
More recently, the power of neuronal oscillations in the beta band was found to be another measure that is predictive of the length of produced intervals.

\citeA{jenkinson_new_2011} suggested that beta power in the basal ganglia provides an index of the likelihood that a voluntary movement will be required, with a low beta power indicating a high probability.
Indeed, \citeA{joundi_driving_2012} found that in a task where participants were instructed to tap along with an external rhythm, beta power was lower during the tapping.
Furthermore, they found that with high tapping rates, beta activity could not fully resynchronize between taps.

Along similar lines, \citeA{bartolo_information_2014} used beta power to predict the interval length in a synchronization-continuation tapping task performed on monkeys.
In this experiment, subjects first tapped along with an externally provided beat (synchronization), and then continued tapping at the same speed without external guidance (continuation).
During the continuation part, longer intervals were associated with higher beta power during the interval, and vice versa.

\citeA{kononowicz_beta_2014} extended these findings to an interval production task in humans, in which participants were instructed to estimate time intervals of 2.5 seconds by pressing a key twice.
In this study, intervals were distinct from each other, providing a clear association between the beta power measured during the interval and the length of the interval itself, and also provided enough time for beta oscillations to fully resynchronize between trials.
As in \citeauthor{bartolo_information_2014}'s study, beta power was postitively correlated with the length of the produced intervals.

% Interval timing -> reaction time
In another area of the field of timing, a large body of research exists that links interval timing to reaction time tasks \cite<e.g.,>{karlin_reaction_1959, drazin_effects_1961, grosjean_timing_2001}.
When participants come to expect the trigger stimulus after a specific time interval, deviations from this interval lead to predictable changes in reaction time.
For example, \citeA{grosjean_timing_2001} found that when the stimulus appears earlier than expected, reaction times will be slower.
Conversely, when it appears later than expected, reaction times will be faster.
An obvious explanation for this effect is that participants build up the readiness to react towards the point in time when they expect the stimulus to occur.
When it appears before they expect it, they are less ready to respond to it, and thus slower.
However, when it appears later than anticipated, the readiness is very high and the reaction comes faster.

% This study
The purpose of the present study is to investigate the link between beta power, interval timing, and reaction time.
As beta power has been found to increase the length of estimated time intervals \cite<e.g.>{kononowicz_beta_2014}, and interval timing has been shown to play a role in reaction time tasks via expectancy effects \cite<e.g.>{grosjean_timing_2001}, we hypothesise in the present study that beta power can be used to predict the expectancy, and hence the response speed, in a reaction time task.

To assess this hypothesis, we performed a simple reaction time task in which participants initiated each trial themselves by pressing a key.
Beta power was measured during the first second of each trial.
The trigger stimulus always occured after a fixed time from the start of the trial during the first 150 trials, in order for subjects to come to expect the stimulus after this interval. In the last 250 trials, the interval was chosen at random from the standard stimulus, and both 100ms longer and shorter intervals.

In line with the findings of \citeA{karlin_reaction_1959, drazin_effects_1961, grosjean_timing_2001}, reaction times were hypothesized to be slower for the shorter interval, and faster on trials with the longer interval.

In addition, we expected reaction times to be faster in trials where beta power is low, because participants hypothetically estimate the interval to end sooner than it actually does, and are thus very ready to respond when the stimulus finally does occur.
Conversely, in trials with high beta power, participants should estimate the stimulus to occur later than it does, and thus not be prepared for it when it actually occurs, leading to a slower reaction time.

\section{Method}
\section{Results}

\begin{table}[ht]
  \caption{Regression of reaction time on length condition and beta power}
  \centering
  \begin{tabular}{lrrrr}
    \toprule
                                                  & B        & SE         & t       & p         \\
    \midrule
    Intercept                                     & -0.0029  & 0.0008     &  -3.42  & $< 0.001$ \\
    Length condition: long                        & -0.0171  & 0.0016     & -10.44  & $< 0.001$ \\
    Length condition: short                       &  0.0259  & 0.0015     &  16.96  & $< 0.001$ \\
    Beta power                                    & -0.0031  & 0.0063     &  -0.49  & $0.622$   \\
    Beta power $\times$ Length condition: long    & -0.0079  & 0.0123     &  -0.64  & $0.521$   \\
    Beta power $\times$ Length condition: short   & -0.0086  & 0.0130     &   0.66  & $0.511$   \\
    \bottomrule
  \end{tabular}
\end{table}

\begin{table}[ht]
  \caption{Descriptive statistics overall as well as by length condition}
  \centering
  \begin{tabular}{lrrrrr}
    \toprule
                             & \multicolumn{2}{l}{Reaction time}              && \multicolumn{2}{l}{Beta}                       \\ \cmidrule{2-3} \cmidrule{5-6}
                             & \multicolumn{1}{l}{M} & \multicolumn{1}{l}{SD} && \multicolumn{1}{l}{M} & \multicolumn{1}{l}{SD} \\
    \midrule
    Overall                  & 0.188                 &  0.067                 && 0.277                 & 0.188                  \\
    Length condition: short  & 0.207                 &  0.070                 && 0.270                 & 0.168                  \\
    Length condition: medium & 0.188                 &  0.061                 && 0.283                 & 0.200                  \\
    Length condition: long   & 0.166                 &  0.070                 && 0.271                 & 0.177                  \\
    \bottomrule
  \end{tabular}
\end{table}

\section{Discussion}
% no separate future research section; include in into rest of text if at all

% talk about both internal/external cues & explicit/implicit timing

% argue with motor readiness about internal/external (for hedderik's experiment: participants can already plan when to press second time directly after the first press)
% argue with different brain regions about explicit/implicit

% explicit timing seems to be needed for motor planning

% talk about bci

% small conclusion in the end is nice
% last sentence should be positive

% no limitations section (findings in foreground, mention limitations only in context of findings)

\bibliographystyle{apacite} % / apalike
% http://www.ctan.org/tex-archive/biblio/bibtex/contrib/apacite/apacite.pdf
\bibliography{/home/max/Dropbox/Psychology/Thesis/latex/clean_references}


\end{document}

\end{document}
