\section{Method}
\label{sec:method}

\subsection{Participants}
X english speaking international students at the University of Groningen (Male: X , female: X) participated in the study, with ages ranging from X to X (MEAN, SD). X participants received partial course credits, while the remaining X were recruited via an advertising on a social network and received €5 for participation.

\subsection{Materials}
A EEG-cap was used to record five locations on the scalp (CZ, FZ, PZ, C3, and C4). TMSI  Furthermore, activity of the eyes (horizontal and vertical) and both mastoids (A1 and A2) were measured. The impedances of all channels were kept below 10 kΩ.
The channels were amplified using a amplifier. The experiment code was written in MATLAB, making use of the Psychtoolbox library (CITATION?).

\subsection{Task \& Stimuli}
Each trial consisted of the following sequence of events. First, a reminder to blink was presented for 1500ms. This was followed by a gray fixation circle. By pressing the spacebar, participants started an interval, indicated by the circle turning white. It then stayed white until the end of the trial. The length of the interval differed between conditions and was not explicitly communicated to the participant. At the end of the interval a target sound (1000 Hz, 0.2 s) was played to which participants had to react as quickly as possible by pressing the spacebar a second time. The reaction time was recorded as the time between the onset of the sound stimulus and the second key press.
Feedback for valid responses was given on a 5-point scale represented by a row of five circles (see Figure x). Responses preceding the sound onset were given feedback in the form of the text “too early” displayed in red.
All visual elements were displayed centrally on the screen to minimize eye-movement artifacts in the EEG signal.

\subsection{Procedure}
The experiment was split into two experimental blocks preceded by five practice trials. Every 105 trials the participants were presented a screen instructing them to take a break.

Block 1 consisted of 150 trials with an invariant interval of 1.8 seconds. The primary purpose of this block was to acquaint the participants with this interval and get them to expect it in later trials.

In the second block of the experiment participants completed 250 trials. Contrary to block 1 the interval length was not fixed, but determined using a BCI setup. Three conditions were used: short (1.7 seconds), standard (1.8 seconds) and long (1.9 seconds).

\subsubsection{EEG}
The EEG signal was measured from the five scalp electrodes CZ, FZ, PZ, C3, and C4, using the average of the mastoid electrodes A1 and A2 as the reference. From the EEG activity during the first second of the interval, the power in the beta band (15-30 Hz) was computed using a fast Fourier transform with Hanning window tapering (see APPENDIX X).

\subsubsection{BCI}
A brain computer interface (BCI) setup was used to facilitate an optimal distribution of length conditions. In each trial, the distribution of beta values from all trials, including the current one, was split into three parts (lower ⅖, middle ⅕, and upper ⅖, representing the beta condition). The beta-power of the participant in the current trial was compared to the current distribution of beta power.
For an average beta value (the middle ⅕ of the distribution) the participants were always shown the standard interval. For a high beta value (upper ⅖ of the distribution) the participants were presented one of the three conditions, determined by randomly drawing (without replacement) from a list that contained five short, five long, and two standard conditions. This procedure ensured that conditions were presented both randomly, but also approximately equally frequent. When the list was empty (i.e. after 12 beta values in the high condition) a new list was created.
A low beta-value (lower ⅖ of the beta distribution) was handled in the same manner as a high beta value, though an independent but identical list was used.
Design

For this experiment a within-subjects design was used. The dependent variable was reaction time to the onset of a tone, presented after a certain interval. The independent variable was beta power, measured from the EEG, and interval length.
